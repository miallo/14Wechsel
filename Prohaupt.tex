% Für Bindekorrektur als optionales Argument "BCORfaktormitmaßeinheit", dann
% sieht auch Option "twoside" vernünftig aus
% Näheres zu "scrartcl" bzw. "scrreprt" und "scrbook" siehe KOMA-Skript Doku
\documentclass[12pt,a4paper,titlepage,headinclude,bibtotoc]{scrartcl}


%---- Allgemeine Layout Einstellungen ------------------------------------------

% Für Kopf und Fußzeilen, siehe auch KOMA-Skript Doku
\usepackage[komastyle]{scrpage2}
\pagestyle{scrheadings}
\setheadsepline{0.5pt}[\color{black}]
\automark[section]{chapter}


%Einstellungen für Figuren- und Tabellenbeschriftungen
\setkomafont{captionlabel}{\sffamily\bfseries}
\setcapindent{0em}


%---- Weitere Pakete -----------------------------------------------------------
% Die Pakete sind alle in der TeX Live Distribution enthalten. Wichtige Adressen
% www.ctan.org, www.dante.de

% Sprachunterstützung
\usepackage[ngerman]{babel}

% Benutzung von Umlauten direkt im Text
% entweder "latin1" oder "utf8"
\usepackage[utf8]{inputenc}

% Pakete mit Mathesymbolen und zur Beseitigung von Schwächen der Mathe-Umgebung
\usepackage{latexsym,exscale,stmaryrd,amssymb,amsmath}

% Weitere Symbole
\usepackage[nointegrals]{wasysym}
\usepackage{eurosym}

% Anderes Literaturverzeichnisformat
%\usepackage[square,sort&compress]{natbib}

% Für Farbe
\usepackage{color}

% Zur Graphikausgabe
%Beipiel: \includegraphics[width=\textwidth]{grafik.png}
\usepackage{graphicx}

% Text umfließt Graphiken und Tabellen
% Beispiel:
% \begin{wrapfigure}[Zeilenanzahl]{"l" oder "r"}{breite}
%   \centering
%   \includegraphics[width=...]{grafik}
%   \caption{Beschriftung} 
%   \label{fig:grafik}
% \end{wrapfigure}
\usepackage{wrapfig}

% Mehrere Abbildungen nebeneinander
% Beispiel:
% \begin{figure}[htb]
%   \centering
%   \subfigure[Beschriftung 1\label{fig:label1}]
%   {\includegraphics[width=0.49\textwidth]{grafik1}}
%   \hfill
%   \subfigure[Beschriftung 2\label{fig:label2}]
%   {\includegraphics[width=0.49\textwidth]{grafik2}}
%   \caption{Beschriftung allgemein}
%   \label{fig:label-gesamt}
% \end{figure}
\usepackage{subfigure}

% Caption neben Abbildung
% Beispiel:
% \sidecaptionvpos{figure}{"c" oder "t" oder "b"}
% \begin{SCfigure}[rel. Breite (normalerweise = 1)][hbt]
%   \centering
%   \includegraphics[width=0.5\textwidth]{grafik.png}
%   \caption{Beschreibung}
%   \label{fig:}
% \end{SCfigure}
\usepackage{sidecap}

% Befehl für "Entspricht"-Zeichen
\newcommand{\corresponds}{\ensuremath{\mathrel{\widehat{=}}}}
% Befehl für Errorfunction
\newcommand{\erf}[1]{\text{ erf}\ensuremath{\left( #1 \right)}}

%Fußnoten zwingend auf diese Seite setzen
\interfootnotelinepenalty=1000

%Für chemische Formeln (von www.dante.de)
%% Anpassung an LaTeX(2e) von Bernd Raichle
\makeatletter
\DeclareRobustCommand{\chemical}[1]{%
  {\(\m@th
   \edef\resetfontdimens{\noexpand\)%
       \fontdimen16\textfont2=\the\fontdimen16\textfont2
       \fontdimen17\textfont2=\the\fontdimen17\textfont2\relax}%
   \fontdimen16\textfont2=2.7pt \fontdimen17\textfont2=2.7pt
   \mathrm{#1}%
   \resetfontdimens}}
\makeatother

%Honecker-Kasten mit $$\shadowbox{$xxxx$}$$
\usepackage{fancybox}

%SI-Package
\usepackage{siunitx}

%keine Einrückung, wenn Latex doppelte Leerzeile
\parindent0pt

%Bibliography \bibliography{literatur} und \cite{gerthsen}
%\usepackage{cite}
\usepackage{babelbib}
\selectbiblanguage{ngerman}

\begin{document}

\begin{titlepage}
\centering
\textsc{\Large Anfängerpraktikum der Fakultät für
  Physik,\\[1.5ex] Universität Göttingen}

\vspace*{4.2cm}

\rule{\textwidth}{1pt}\\[0.5cm]
{\huge \bfseries
  Wechselstromwiderstände\\[1.5ex]
  Protokoll}\\[0.5cm]
\rule{\textwidth}{1pt}

\vspace*{2.5cm}

\begin{Large}
\begin{tabular}{ll}
Praktikant: &  Michael Lohmann\\
 Versuchspartner &  Felix Kurtz\\
 E-Mail: & m.lohmann@stud.uni-goettingen.de\\
 Betreuer: & Björn Klaas\\
 Versuchsdatum: & 08.09.2014\\
\end{tabular}
\end{Large}

\vspace*{0.8cm}

\begin{Large}
\fbox{
  \begin{minipage}[t][2.5cm][t]{6cm} 
Eingegangen am:
  \end{minipage}
}
\end{Large}

\end{titlepage}

\tableofcontents

\newpage

\section{Einleitung}
\label{sec:einleitung}
Wechselströme spielen in der modernen Energieversorgung eine zentrale Rolle.
Um so wichtiger ist es für die Effizienz, die genauen Eigenschaften von \emph{Wechselstrom-Widerständen} zu kennen.
Dies soll in diesem Versuch erziehlt werden.
\section{Theorie}
\label{sec:theorie}
\subsection{Wechselspannungen und -ströme}
Unter einer Wechselspannung versteht man eine Spannung, die einen zeitlichen Verlauf von $U(t)= U_0\cdot\sin(\omega t)$ besitzt.
Da hierbei die maximale Spannung $U_0$ nur den kürzesten Teil der Zeit anliegt, ist die Definition eines \emph{Effektivwertes} sinnvoll.
Diese ist so definiert, dass eine Gleichspannung diesen Wertes die selbe Leistung erbringt, nämlich für $\sin$-förmige Spannungen:
\begin{align*}
U_\text{eff}=\frac{U_0}{\sqrt2}\quad .
\end{align*}
Die in europäischen Haushalten übliche Wechselspannung von $U_\text{eff}=220\,\si\volt$ besitzt so eine Maximalspannung von $U_0=\sqrt2\cdot U_\text{eff}=311\,\si\volt$.\\

Der fließende Strom $I$ besitzt nun die Form
\begin{align*}
I=I_0\cdot\sin(\omega t+\varphi)
\end{align*}
wobei $\varphi$ die Phasenverschiebung ist.
Diese hängt ab von der jeweiligen Schaltung ab und wird in den Kapiteln \ref{sec:RLC} und \ref{sec:LC} behandelt.

\subsection{Induktivität}
Nach \cite[S. 313]{griffith} gilt für eine Spule
\begin{align}
U_\text{Ind}=-U_L=-L\frac{dI}{dt}\qquad .
\end{align}
Hierbei bezeichnet $L$ die Induktivität, welche von der Spulengeometrie abhängt.
Nimmt man nun wieder die Sinus-Spannung an, so folgt:
\begin{align*}
U_0\cdot\sin(\omega t)&=L\frac{dI}{dt}\\
\Leftrightarrow I(t)&=U_0\frac{1}{\omega L}\cos(\omega t)\qquad .
\end{align*}
Dies bedeutet, dass die Spannung dem Strom für eine Schaltung, welche nur aus einer Spule bestehent, um $\varphi=\frac{\pi}{2}$ voraus eilt.

Man kann nun den \emph{Blindwiderstand} für eine Spule definieren:
\begin{align}
X_L:=\frac{U(t)}{I(t)}=L\omega\quad .
\end{align}



\subsection{Kapazität}
Für einen Kondensator gilt nach \cite[S. 822]{giancoli}:
\begin{align*}
Q&=CU=CU_0\sin(\omega t)\\
\Rightarrow I(t)&=\dot Q=CU_0\omega\cos(\omega t)\quad .
\end{align*}
Dies bedeutet, dass der Strom der Spannung um $\varphi=\frac{\pi}{2}$ voraus eilt.

Auch für einen Kondensator kann man so einen \emph{Blindwiederstand} definieren:
\begin{align}
X_C:=\frac{U(t)}{I(t)}=-\frac{1}{\omega C}
\end{align}

\subsection{$RLC$-Serienschaltung}
\label{sec:RLC}
\begin{figure}[!h]
\centering
\includegraphics[width=0.6\linewidth]{serie}
\caption{Schaltplan der Serienschaltung von \cite[4.10.2014, 15:30]{LP14}.}
\label{fig:serienschaltung}
\end{figure}
Ein $RLC$-Serienschaltkreis wie in Abb. \ref{fig:serienschaltung} muss nun aus einer Zusammensetzung der beiden erfolgen.
Dafür definiert man die \emph{Impedanz}:
\begin{align}
Z:=\left|R+i(X_C+X_L)\right|=\left|R+i\left( \omega L-\frac{1}{\omega C} \right)\right|=\sqrt{R^2+(X_L+X_C)^2}\quad .
\end{align}
Die Phasenverschiebung lässt sich (wie in Abb. \ref{fig:zeiger} zu erkennen und nach \cite[S. 1042]{giancoli}) durch
\begin{align}
\varphi=\arctan\left( \frac{X_L+X_C}{R} \right)
\end{align}
berechnen.

\subsection{$LC$-Parallelschaltung}
\label{sec:LC}
\begin{figure}[!h]
\centering
\includegraphics[width=0.6\linewidth]{parallel}
\caption{Schaltplan der Parallelschaltung von \cite[4.10.2014, 15:30]{LP14}.}
\label{fig:parallel}
\end{figure}

\subsection{Impedanz}
\begin{figure}[!h]
\centering
\input{zeiger.pdf_tex}
\caption{Zeigerdiagramm, welches die Realteile der Spannung gegen ihren Imaginärteil aufträgt.}
\label{fig:zeiger}
\end{figure}

\section{Durchführung}
\label{sec:durchfuehrung}
Der Aufbau besteht aus einem Frequenzgenerator, welcher einem veränderlichen Stromkreis aus Widerstand, Kondensator und Luftspule Spannung bereitstellt.
Die verschiedenen Parameter Ausgangsspannung $U$, Spannung an Widerstand und Spule $U_{L+R}$, Spannung am Kondensator $U_C$ und Gesamtstrom $I$ werden mit einem Oszilloskop bzw. Spannungs- und Strom-Messgeräten vermessen.\\

Zunächst baut man einen Serienschaltkreis aus allen Bauteilen auf.
Das Oszilloskop wird zur Bestimmung der Phasenverschiebung einerseits zur Vermessung der Ausgangsspannung $U$ und andererseits zum bestimmen des Stroms mit einer Messzange verwendet.
Es kann nun die beiden Kurven mit Hilfe des Mathe-Modus direkt auf deren Phasenverschiebung hin auswerten.
Damit dies zuverlässig geschieht, ist darauf zu achten, dass die jeweiligen $y$-Achsen so gewählt sind, dass die Kurven ungefähr die selben Ausschläge zeigen.
Auch muss mehr als eine Periode angezeigt werden.

Alle gemessenen Parameter sollen nun für möglichst viele verschiedene Frequenzen $f$ aufgezeichnet werden.
Dabei ist der Resonanzbereich besonders genau zu untersuchen.\\

Im zweiten Versuchsteil soll ein Parallelkreis aus Kondensator und Spule vermessen werden.
In dieser Messung sollen die Spannung $U$ und der Gesamtstrom $I$ für verschiedene Frequenzen ausgewertet werden.
Auch hier soll die Resonanzstelle wieder besonders genau untersucht werden.\\

Für die Auswertung werden abschließend die Daten der einzelnen Bauteile aufgezeichnet.
Dies sind:
\begin{itemize}
\item Einzelner ohmscher Widerstand $R_\Omega$
\item Ohmscher Widerstand der Spule $R_L$
\item Innenwiderstand des Amperemeters $R_A$
\item Kapazität des Kondensators $C$.
\end{itemize}

Während der Messungen ist darauf zu achten, dass die hier verwendeten Spannungen \emph{tödlich} sein können und dass deshalb auf keinen Fall blanke Kabelenden herumliegen dürfen.
Auch muss vor jeden Änderungen am Aufbau sichergestellt werden, dass die Spannung abgeschaltet ist.


\section{Auswertung}
\label{sec:auswertung}
\subsection{Widerstand und Spule in Reihe}
\begin{figure}[!htb]
	\centering
	\input{messung1.tex}
	\caption{Quadrat der Impedanz als Funktion der Kreisfrequenz.}
	\label{fig:messung1}
\end{figure}

\begin{align}
	L&=(386.3\pm 0.6)\,\si{\milli\henry}\\
	R_\text{ges}&=(77.3 \pm 1.1)\,\si{\ohm}
\end{align}
\subsection{$RLC$-Serienschaltung}
\begin{figure}[!htb]
	\centering
	\input{messung2.tex}
	\caption{Impedanz des Serienresonanzkreis als Funktion der Kreisfrequenz.}
	\label{fig:messung2}
\end{figure}
Aus
\begin{align}
	R &= (80.9 \pm 0.5)\,\si{\ohm}\\
	L &= (386.1 \pm 1.0)\,\si{\milli\henry}\\
	C &= (1.799 \pm 0.005)\,\si{\micro\farad}
\end{align}
Mittelwerte aus allen Daten:
\begin{align}
\overline L&=(386.2 \pm 0.6)\si{\milli\henry}\\
\overline R&=(80.2836 \pm 0.455183)\,\si{\ohm}
\end{align}
\begin{align}
\omega_\text{LC}&=\frac{1}{\sqrt{LC}}\\
\sigma_{\omega_\text{LC}}&=\frac{\sqrt{\frac{\sigma_{L}^{2}}{L^{2}} + \frac{\sigma_{C}^{2}}{C^{2}}}}{2 \cdot \sqrt{C} \cdot \sqrt{L}}\\
\omega_\text{LC}&=(1199.9 \pm 2.3)\,\si{\hertz}
\end{align}
\begin{figure}[!htb]
	\centering
	% GNUPLOT: LaTeX picture with Postscript
\begingroup
  \makeatletter
  \providecommand\color[2][]{%
    \GenericError{(gnuplot) \space\space\space\@spaces}{%
      Package color not loaded in conjunction with
      terminal option `colourtext'%
    }{See the gnuplot documentation for explanation.%
    }{Either use 'blacktext' in gnuplot or load the package
      color.sty in LaTeX.}%
    \renewcommand\color[2][]{}%
  }%
  \providecommand\includegraphics[2][]{%
    \GenericError{(gnuplot) \space\space\space\@spaces}{%
      Package graphicx or graphics not loaded%
    }{See the gnuplot documentation for explanation.%
    }{The gnuplot epslatex terminal needs graphicx.sty or graphics.sty.}%
    \renewcommand\includegraphics[2][]{}%
  }%
  \providecommand\rotatebox[2]{#2}%
  \@ifundefined{ifGPcolor}{%
    \newif\ifGPcolor
    \GPcolortrue
  }{}%
  \@ifundefined{ifGPblacktext}{%
    \newif\ifGPblacktext
    \GPblacktexttrue
  }{}%
  % define a \g@addto@macro without @ in the name:
  \let\gplgaddtomacro\g@addto@macro
  % define empty templates for all commands taking text:
  \gdef\gplbacktext{}%
  \gdef\gplfronttext{}%
  \makeatother
  \ifGPblacktext
    % no textcolor at all
    \def\colorrgb#1{}%
    \def\colorgray#1{}%
  \else
    % gray or color?
    \ifGPcolor
      \def\colorrgb#1{\color[rgb]{#1}}%
      \def\colorgray#1{\color[gray]{#1}}%
      \expandafter\def\csname LTw\endcsname{\color{white}}%
      \expandafter\def\csname LTb\endcsname{\color{black}}%
      \expandafter\def\csname LTa\endcsname{\color{black}}%
      \expandafter\def\csname LT0\endcsname{\color[rgb]{1,0,0}}%
      \expandafter\def\csname LT1\endcsname{\color[rgb]{0,1,0}}%
      \expandafter\def\csname LT2\endcsname{\color[rgb]{0,0,1}}%
      \expandafter\def\csname LT3\endcsname{\color[rgb]{1,0,1}}%
      \expandafter\def\csname LT4\endcsname{\color[rgb]{0,1,1}}%
      \expandafter\def\csname LT5\endcsname{\color[rgb]{1,1,0}}%
      \expandafter\def\csname LT6\endcsname{\color[rgb]{0,0,0}}%
      \expandafter\def\csname LT7\endcsname{\color[rgb]{1,0.3,0}}%
      \expandafter\def\csname LT8\endcsname{\color[rgb]{0.5,0.5,0.5}}%
    \else
      % gray
      \def\colorrgb#1{\color{black}}%
      \def\colorgray#1{\color[gray]{#1}}%
      \expandafter\def\csname LTw\endcsname{\color{white}}%
      \expandafter\def\csname LTb\endcsname{\color{black}}%
      \expandafter\def\csname LTa\endcsname{\color{black}}%
      \expandafter\def\csname LT0\endcsname{\color{black}}%
      \expandafter\def\csname LT1\endcsname{\color{black}}%
      \expandafter\def\csname LT2\endcsname{\color{black}}%
      \expandafter\def\csname LT3\endcsname{\color{black}}%
      \expandafter\def\csname LT4\endcsname{\color{black}}%
      \expandafter\def\csname LT5\endcsname{\color{black}}%
      \expandafter\def\csname LT6\endcsname{\color{black}}%
      \expandafter\def\csname LT7\endcsname{\color{black}}%
      \expandafter\def\csname LT8\endcsname{\color{black}}%
    \fi
  \fi
  \setlength{\unitlength}{0.0500bp}%
  \begin{picture}(7200.00,5040.00)%
    \gplgaddtomacro\gplbacktext{%
      \csname LTb\endcsname%
      \put(1254,704){\makebox(0,0)[r]{\strut{}-120}}%
      \put(1254,1074){\makebox(0,0)[r]{\strut{}-100}}%
      \put(1254,1444){\makebox(0,0)[r]{\strut{}-80}}%
      \put(1254,1815){\makebox(0,0)[r]{\strut{}-60}}%
      \put(1254,2185){\makebox(0,0)[r]{\strut{}-40}}%
      \put(1254,2555){\makebox(0,0)[r]{\strut{}-20}}%
      \put(1254,2925){\makebox(0,0)[r]{\strut{} 0}}%
      \put(1254,3295){\makebox(0,0)[r]{\strut{} 20}}%
      \put(1254,3665){\makebox(0,0)[r]{\strut{} 40}}%
      \put(1254,4036){\makebox(0,0)[r]{\strut{} 60}}%
      \put(1254,4406){\makebox(0,0)[r]{\strut{} 80}}%
      \put(1254,4776){\makebox(0,0)[r]{\strut{} 100}}%
      \put(1386,484){\makebox(0,0){\strut{} 0}}%
      \put(2083,484){\makebox(0,0){\strut{} 500}}%
      \put(2779,484){\makebox(0,0){\strut{} 1000}}%
      \put(3476,484){\makebox(0,0){\strut{} 1500}}%
      \put(4172,484){\makebox(0,0){\strut{} 2000}}%
      \put(4869,484){\makebox(0,0){\strut{} 2500}}%
      \put(5565,484){\makebox(0,0){\strut{} 3000}}%
      \put(6262,484){\makebox(0,0){\strut{} 3500}}%
      \put(6958,484){\makebox(0,0){\strut{} 4000}}%
      \put(484,2740){\rotatebox{90}{\makebox(0,0){\strut{}$\varphi$ [Grad]}}}%
      \put(4172,154){\makebox(0,0){\strut{}$\omega$ [Hz]}}%
    }%
    \gplgaddtomacro\gplfronttext{%
      \csname LTb\endcsname%
      \put(2706,4603){\makebox(0,0)[r]{\strut{}Messwerte}}%
    }%
    \gplbacktext
    \put(0,0){\includegraphics{phase}}%
    \gplfronttext
  \end{picture}%
\endgroup

	\caption{Phasenverschiebung des Serienresonanzkreises.}
	\label{fig:phase}
\end{figure}

\begin{align}
\omega_\text{Phase}&=- \frac{b}{m}\\
\sigma_{\omega_\text{Phase}}&=\frac{1}{m^{2}} \cdot \sqrt{b^{2} \cdot \sigma_{m}^{2} + m^{2} \cdot \sigma_{b}^{2}}\\
\omega_\text{Phase}&=(1200 \pm 120)\,\si\hertz
\end{align}
\begin{figure}[!htb]
	\centering
	\input{spannungen.tex}
	\caption{Teilspannungen des Serienresonanzkreises.}
	\label{fig:teilU}
\end{figure}
\subsection{Parallelkreis}
Aus Fit von Messung 3:
\begin{align}
R &= (68\pm 5)\si{\kilo\ohm}\\
L &= (370 \pm 10)\si{\milli\henry}\\
C &= (1.88  \pm 0.05) \si{\micro\farad}
\end{align}
\begin{figure}[!htb]
	\centering
	\input{messung3.tex}
	\caption{Impedanz des Parallelkreises als Funktion der Kreisfrequenz.}
	\label{fig:messung3}
\end{figure}

\section{Diskussion}
\label{sec:diskussion}

\bibliography{literatur}
\bibliographystyle{babalpha}
\end{document}
